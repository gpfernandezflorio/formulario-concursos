% Divu
\begin{itemize}[leftmargin=0.5cm]
\WorkEntry{\textbf{Divulgador Departamental}}
{FCEyN - UBA}
{Durante 2017 y 2018}{}
\end{itemize}

\begin{enumerate}[leftmargin=0.8cm]

  \item[a)]{Proyectos de Extensión actuales y anteriores

    \begin{itemize}[leftmargin=0.2cm]

      \item[i)]{realizados en el ámbito de las Universidades Nacionales.

        \begin{itemize}[leftmargin=0.2cm]

          % Talleres ComCom
          \WorkEntry{Participación en los \textbf{Talleres ComCom}\footnote{Talleres para estudiantes de la carrera
            de computación y otras carreras de la facultad.}}
          {FCEyN - UBA}
          {Desde 2018.}
          {Armado y dictado de distintos talleres. \\
          Hasta ahora sólo pudimos dictar el \textbf{Taller Introductorio de Linux}, de 2 clases de 2 horas, pero tenemos otros armados.}

          % Robótica en la Escuela
          \WorkEntry{Participación en el proyecto \textbf{Robótica en la Escuela}\footnote{A cargo del
            Departamento de Computación de la facultad, como parte del prograba \textbf{UBANEX}. Consiste
            en un proyecto para promover la enseñanza de la programación y la robótica en las escuelas,
            mediante el uso de un simulador online (\url{http://www.roboticaenlaescuela.dc.uba.ar/}).}}
          {FCEyN - UBA}
          {Desde 2018.}
          {Diseñé en Blender la nueva versión del robot que utilizaremos en los talleres y cuyas
            especificaciones se compartirán a las escuelas para que puedan armar su propio robot.
            También armé nuevos mapas para ser utilizados en los talleres.}

          % Bacterminador
          \WorkEntry{Participación en el proyecto \textbf{Bacterminador}\footnote{A cargo del Departamento
            de Química Orgánica de la facultad, como parte del prograba \textbf{UBANEX}. Consiste en un juego
            pensado para ser utilizado por docentes de escuela primaria para enseñar conceptos básicos de higiene
            y microbiología de alimentos. Detalles del proyecto: \url{http://bacterminador.com.ar/sobre-el-proyecto/}}}
          {FCEyN - UBA}
          {Desde 2017.}
          {Además de algunas mejoras menores, rehice la aplicación en el motor gráfico Godot para
            que sea más fácil de mantener que la versión anterior (desarrollada en Construct 2) y
            para poder exportarla a diferentes plataformas, entre ellas, web, Windows, Linux, Mac y Android.}

        \end{itemize}
      }

    \end{itemize}
  }

  \item[b)]{Actividades

    \begin{itemize}[leftmargin=0.2cm]

      \item[i)]{de divulgación científica.

        \begin{itemize}[leftmargin=0.2cm]

          % Feria de Profesiones septiembre 2019
          \WorkEntry{Participación en la jornada \textbf{¿Para qué seguir estudiando?}\footnote{Feria
            de profesiones en la \textbf{Escuela Técnica de la UBA} en Villa Lugano, como parte del
            programa de tutorías \textbf{Universitarios por más Universitarios} (\url{http://www.uba.ar/contenido/448}).}\ 2019}
          {Escuela Técnica - UBA (Sede Lugano)}
          {28 de Septiembre de 2019, de 9hs a 14hs.}
          {\emph{Expositor} en las distintas estaciones demostrativas.}

          %% CP 2 Junio 2019
          \WorkEntry{Participación en \textbf{Ciencia Paliza 2}}
          {Parque Rivadavia}
          {1 de Junio de 2019, de 15hs a 18hs.}
          {\emph{Expositor} en las distintas estaciones demostrativas.}

          % Ciencia Palooza marzo 2018
          \WorkEntry{Participación en la jornada \textbf{Un Festival de Ciencia (ex Ciencia Palooza)}\footnote{
            Festival organizado por Expedición Ciencia (\url{http://expedicionciencia.org.ar/}). Detalles del
            evento: \url{http://www.redciteco.org/actividades/eventos/20-cienciapalooza}}}
          {Centro Cultural Konex}
          {29 de Marzo de de 2018, de 18hs a 2hs.}
          {\emph{Expositor} en el stand de Inteligencia Artificial. \\
          También colaboré con el armado de material.}

          % NdlM noviembre 2017
          \WorkEntry{Participación en \textbf{La Noche de la Ciencia} 2017}
          {La Noche de los Museos}
          {4 de Noviembre de 2017, de 19hs a 3hs, durante toda la jornada.}
          {\emph{Expositor} en las distintas estaciones demostrativas.}

          % SdC 2015
          \WorkEntry{Participación en la \textbf{Semana de la Computación} 2015}
          {}{16, 17 y 18 de Junio de 2015, los tres días durante toda la jornada.}
          {\emph{Colaborador externo} en el stand \textbf{Búsqueda Binaria}. \\
          También colaboré con el material utilizado en el stand.}

        \end{itemize}

      }

      \item[ii)]{de articulación con otros niveles educativos.

        \begin{itemize}[leftmargin=0.2cm]

          % Cx1D diciembre 2018
          \WorkEntry{Participación en la jornada \textbf{Científicos por un Día}\footnote{
            \url{http://www.fcen.uba.ar/dov/cientificos_por_un_dia/cientificos_por_un_dia.htm}}\ 2018}
          {}{14 de Diciembre de 2018, de 8:30hs a 18hs.}
          {\emph{Docente}. Participé tanto en el dictado como en el armado de la clase.}

          % EVE agosto 2018
          \WorkEntry{Participación en el \textbf{Taller de Programación y Robótica}, como parte del programa
            \textbf{La Escuela Viene a Exactas}\footnote{
            \url{https://exactas.uba.ar/extension/comunicacion-publica-de-la-ciencia/exactas-va-a-la-escuela/}}}
          {}{10 de Agosto de 2018, de 10hs a 12hs.}
          {\emph{Docente}. Participé en el armado de la clase.}

        \end{itemize}

      }

    \end{itemize}
  }

  \item[c)]{Publicaciones
    \\ No corresponde
  }

  \item[d)]{Presentaciones de proyectos de extensión en congresos, jornadas y otros encuentros de la especialidad.
    \\ No corresponde
  }

  \item[e)]{Otras actividades de extensión no contempladas en los puntos anteriores.
    \\ No corresponde
  }

\end{enumerate}
